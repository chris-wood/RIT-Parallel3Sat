\documentclass[paper=a4, fontsize=10pt]{scrartcl} % A4 paper and 11pt font size
\usepackage[left=2cm,top=1cm,right=2cm,nohead,nofoot]{geometry}
\usepackage[T1]{fontenc} % Use 8-bit encoding that has 256 glyphs
\usepackage{fourier} % Use the Adobe Utopia font for the document - comment this line to return to the LaTeX default
\usepackage[english]{babel} % English language/hyphenation
\usepackage{amsmath,amsfonts,amsthm} % Math packages
\usepackage{listings}

\usepackage{lipsum} % Used for inserting dummy 'Lorem ipsum' text into the template

\usepackage{sectsty} % Allows customizing section commands
%\allsectionsfont{\normalfont\scshape} % Make all sections centered, the default font and small caps

\usepackage{fancyhdr} % Custom headers and footers
\pagestyle{fancyplain} % Makes all pages in the document conform to the custom headers and footers
\fancyhead{} % No page header - if you want one, create it in the same way as the footers below
\fancyfoot[L]{} % Empty left footer
\fancyfoot[C]{} % Empty center footer
\fancyfoot[R]{\thepage} % Page numbering for right footer
\renewcommand{\headrulewidth}{0pt} % Remove header underlines
\renewcommand{\footrulewidth}{0pt} % Remove footer underlines

\usepackage[compact]{titlesec}
\titlespacing{\section}{0pt}{*0}{*0}
\titlespacing{\subsection}{0pt}{*0}{*0}
\titlespacing{\subsubsection}{0pt}{*0}{*0}

\setlength{\parskip}{\baselineskip}%
\setlength{\parsep}{0pt}
\setlength{\headsep}{0pt}
\setlength{\topskip}{0pt}
\setlength{\topmargin}{0pt}
\setlength{\topsep}{0pt}
\setlength{\partopsep}{0pt}

%\numberwithin{equation}{section} % Number equations within sections (i.e. 1.1, 1.2, 2.1, 2.2 instead of 1, 2, 3, 4)
%\numberwithin{figure}{section} % Number figures within sections (i.e. 1.1, 1.2, 2.1, 2.2 instead of 1, 2, 3, 4)
%\numberwithin{table}{section} % Number tables within sections (i.e. 1.1, 1.2, 2.1, 2.2 instead of 1, 2, 3, 4)

\setlength\parindent{0pt} % Removes all indentation from paragraphs - comment this line for an assignment with lots of text

%----------------------------------------------------------------------------------------
%	TITLE SECTION
%----------------------------------------------------------------------------------------

\newcommand{\horrule}[1]{\rule{\linewidth}{#1}} % Create horizontal rule command with 1 argument of height

\title{	
\normalfont \normalsize 
\textsc{Department of Computer Science, Rochester Institute of Technology} \\ % Your university, school and/or department name(s)
\textsc{Parallel Computing I, Graduate Team Project} 
\horrule{2pt} \\[0.4cm] % Thin top horizontal rule
\huge Exhaustive Search Algorithms for $3$-$SAT$ \\ 
\Large Team Formation Document \\
\horrule{2pt}
}
\author{Christopher Wood, Eitan Romanoff, Ankur Bajoria } % Your name
\date{\large \today} % Today's date or a custom date

\begin{document}

\maketitle % Print the title

\section{Problem Description}
For our project we propose 3-SAT (or, more formally, 3-CNF-SAT), which 
is an NP-complete problem \cite{algs}. This is a decision problem in which takes 
as input a 3-CNF Boolean formula and returns YES if the formula is 
satisfiable, and NO otherwise. A 3-CNF formula, more formally known 
as a Boolean formula in 3-conjunctive normal form, is expressed as the 
Boolean AND of arbitrarily many clauses, where each clause is the Boolean 
OR of three literals, which is a Boolean variable or its negation. Such a 
Boolean formula is said to be satisfiable if and only if there exists an 
assignment of truth values to the variables such that substituting them 
into the literals of the formula will cause it to evaluate to true (or 1). 
Expressed as a formal language, we have that $3-SAT = \{\langle \phi \rangle : \phi \text{ is satisfiable } \}$.

\section{Exhaustive Search Algorithms for $3$-$SAT$}
An exhaustive search algorithm for solving the 3-SAT problem iterates over 
every combination of variable truth values (of which there exists $2^n$ total 
for n variables), substitutes (or assigns) them to the appropriate literal in each 
clause, and then evaluates the formula to determine if it is satisfiable. If for 
every possible variable truth assignment the formula does not evaluate to 
true (or 1), then the exhaustive 3-SAT algorithm returns NO. Otherwise, 
some satisfiable truth value assignment must exist, and so the 3-SAT algorithm 
returns YES. Formally, the exhaustive search algorithm is defined as follows:

\section{Programs and Performance Metrics}
The sequential and parallel programs we will deliver will take as input 
the 3-CNF formula, encoded using the DIMACS CNF format, 
and output a single Boolean truth value indicating whether or not the 
formula is satisfiable. The 3-CNF formula will be entered at the command 
line or it will be read from a file to facilitate our experiments. Based on 
the 3-SAT problem, each clause must have exactly three literals, so our 
program will enable the number of clauses and the number of variables 
to be parameters defined in the DIMACS CNF format. An example the 
3-CNF formula $(x_1 \lor x_2 \lor x_3)$ problem encoded using DIMACS is shown below.
\begin{center}
\begin{lstlisting}
p cnf 3 1
1 2 -3 0
\end{lstlisting}
\end{center}
There are two main parts of the exhaustive search algorithm for $3$-$SAT$: 
reading in the 3-CNF formula and setting up the appropriate data structures, 
and traversing and substituting all possible combinations of variable truth 
assignments into the 3-CNF formula to check for satisfiability. Both the 
sequential and parallel programs will share the first part so as to set up 
the globally accessible data structures containing the 3-CNF formula. 
This is a fixed amount of sequential overhead that must occur before any 
satisfiability checks can begin. 

The second part of the program can be done in parallel. Formally, the 
second part is an instance of an agenda parallel problem, where each 
task attempts to satisfy a specific truth value assignment. This is because 
we are not concerned with all satisfiability results, we are only concerned 
with the answer to the question for all truth value assignments, ``does this 
truth value assignment satisfy the 3-CNF formula?'' Therefore, since we 
are only seeking the output of one task that answers this question, and 
there are no sequential dependencies between each task, we can divvy 
up the execution of these $2^n$ tasks among $2^n$ virtual processors. When 
implemented, we will clump the execution of many tasks together on a single 
processor because it is unlikely that we will have $2^n$ processors available. 

Since $3$-$SAT$ is both an interesting problem in academia and often arises in 
the industry, we will be measuring the metrics of speedup and speedup, as 
well as the efficiency and sizeup efficiency. To acquire these metrics and 
attempt to model our problem with Amdal's and Gustafson's Law, we will 
measure the execution time of the sequential and parallelizable parts of 
the sequential and parallel programs. This will enable us to calculate the 
total execution time and sequential fraction of the program. Using these 
measurements, along with the problem size N (number of variables) and 
number of processors K, we can evaluate the aforementioned metrics. 

\section{Literature Survey and Additional Resources}
As part of the graduate requirement, we will analyze \cite{paper1}, \cite{paper2},
and \cite{paper3}. 

%%%%% REFERENCES %%%%%

\begin{thebibliography}{9}

\bibitem{algs} Gormen, Thomas H., Charles E. Leiserson, Ronald L. Rivest, and Clifford Stein. Introduction to Algorithms. MIT Press 44 (1990), 97-138.

\bibitem{paper1} Meyer, Quirin, et al. 3-SAT on CUDA: Towards a massively parallel SAT solver. \emph{High Performance Computing and Simulation (HPCS)}, 2010 International Conference on. IEEE, 2010.

\bibitem{paper2} Hamadi, Youssef, Said Jabbour, and Lakhdar Sais. ManySAT: A Parallel SAT Solver. \emph{Journal on Satisfiability, Boolean Modeling and Computation} 6.4 (2009): 245-262.

\bibitem{paper3} Zhang, Hantao, Maria Paola Bonacina, and Jieh Hsiang. PSATO: A Distributed Propositional Prover and its Application to Quasigroup Problems. \emph{Journal of Symbolic Computation} 21.4 (1996): 543-560.

\end{thebibliography}

\end{document}